\chapter{Fundamentals}
In this chapter, the technical basics of this thesis are presented, initially, an introduction to virtual machines and penetration testing is given, followed by some explanation on python and YAML files. Also, the concepts of validation and templating are briefly overviewed.

\section{Virtual Machines}
A virtual machine (VM) is a virtual environment that works like a computer system with its own resources, like CPU, memory, and storage, created on an actual physical hardware system. With the help of a software called  “hypervisor”, the machine’s resources get separated from the hardware so they can be provided in the right manner to be used by the VM.\\
The physical machines, ones equipped with a hypervisor, are called host machines (host), while the many VMs that utilize its resources are guest machines (guest). The hypervisor treats the host’s resources as a pool of resources that can be simply distributed and relocated between existing guests as well as new virtual machines. VMs are also isolated from the rest of the system, and multiple of them can co-exist on a single physical piece of hardware. They can be dynamically relocated between host servers depending on demand.\\
One of the advantages of virtual machines is allowing numerous operating systems to run on a single computer at the same time, and each operating system runs as if it’s running on the host hardware, thus the user experience within the VM is almost identical to that of a real-time operating system experience running on a physical machine\cite{vm_redhat}.\\
\\
This allows penetration testers to apply their knowledge on disposable sandboxes that are as real as host systems, with no worry of potentially damaging hardware or harming people/organizations.

\subsection{Snapshots}
A VM snapshot saves the state and data of a virtual machine at a specific time. The state includes the VM power state ( powered-off, running, aborted), and the data includes all data stored on the VM (memory, disk)\cite{vmware}.\\
The snapshot acts as a copy of the virtual machine and can be used to create multiple instances of the same VM or to restore the VM to a former state. Snapshots can be very useful in development and testing environments, where testing several code changes is needed to have a safe rollback point\cite{carklin_2021}.\\
In the scope of this project, snapshots were used to roll back to a former VM state, in which the modules were not yet installed. It is not necessary to reinstall the operating system on the VM, instead, the VM can be restored to a fresh OS install before starting to run the modules install process.


\section{Penetration Testing}
A penetration test, or a pen test, is an attempt to safely expose and secure an environment by exposing known security issues. These issues can be found in operating systems, applications, or improper configurations. They are also used to validate the effectiveness of various security policies and defensive mechanisms.\\
Penetration testing is mostly performed using automated or manual technologies to systematically exploit servers, endpoints, web apps, and other possible exposures.\\
After successfully exposing a particular system, testers may try to launch other attacks on other internal resources in order to gain deeper access to that system via privilege escalation.\\
Penetration testing is a process utilized by various companies and organizations to identify and assess the security risks associated with various systems and networks. The results of the tests are typically presented to the organizations involved in the security operations and management to be resolved\cite{pentest}.\\
This work aims to make the learning of penetration testing simpler by automating the process of setting up a testing environment.\\


\section{Python}
Python is a high-level programming language that has a variety of object-oriented features. Its flexible and high-level structure makes it very attractive for developing rapid application development. Its simple and easy-to-learn syntax helps minimize program maintenance.\\
The rapid edit-test-debug cycle of Python makes it very easy to debug programs. When an error occurs, the interpreter prints a stacktrace, which tells the program which of the available exceptions has been encountered.\\
The source level debugger simplifies the debugging process by allowing the program to inspect and evaluate the code at a time\cite{python.org}.\\
Python was the language of choice for this software, for the fact that it has most of the packages that are necessary for this work, in addition to its automation capabilities and ease of implementation.\\
Python 3.8 is the language and version of choice for the implementation of this work.


\section{Metadata}
Metadata is a type of structured reference data that contains the attributes of an information source. It is often referred to as data that describes other data.\\
Meta is a prefix that simply means "an underlying description or definition of data." Metadata helps users easily find, use, reuse certain instances of data. For example, author, date created and file size are basic document file metadata.
Metadata can be created in various ways, such as manual or automated. The former can likely be more accurate, allowing the user to input any information that they feel is relevant to the file\cite{Kranz2021Jul}.\\
Manual creation of metadata is the relevant type for this project. Modules creators must write a metadata YAML file, that describes the most crucial information to run this module.
This metadata file can hold the name, version, and different dependencies, which this module provides or needs. Although, most importantly are the configurations details.\\
With metadata describing the relevant information in the module, it makes reading this module to run it more accurate and error-free.


\section{YAML}
YAML is a data serialization language that is often used to create configuration files,
It stands for "yet another markup language" and evolved into "ain’t markup language",  which highlights that YAML is for data and not for documents. It is also easy to understand and is human-readable.\\
YAML is a superset of JSON, so JSON files are valid in YAML, but it uses Python-style indentation to indicate nesting, as there are no usual format symbols, such as braces, square brackets, YAML files use a .yml or .yaml extension.\\
The structure of a YAML file is a map or a list.
Mappings allow you to group key-value pairs into distinct values. Order is not relevant, and each key must be unique. A map needs to be resolved before it can be closed. A new map can then be created by either creating an adjacent map or increasing the indentation level.\\
A list sequence is a type of object that contains values in an order. It can contain multiple items, and starts with a dash (-) and a space, while indentation separates it from the parent.
Naturally, YAML also contains scalars that can be used as values such as strings, integers, or booleans\cite{yaml_redhat}. The following is an example of YAML syntax:

\begin{lstlisting}[caption=YAML example, style=pythonstyle]
---
name: max
enrolled: True
languages:
  - english
  - german
marks:
  - programming: failed
  - math: 1.0

\end{lstlisting}
YAML is the dominant file type for writing configuration files and metadata, it has the benefit of easier human readability, which helps module creators to step into writing metadata rapidly and comfortably.


\section{Validation}
In basic automated systems, data is entered with minimal or even no human supervision. Therefore, it is essential to ensure that the data that enters the system is correct and meets the needed values and standards\cite{BibEntry2021Jul}.\\
Data validation is an automated check, which ensures that the values of the input data are logical and acceptable. However, it does not ensure the correctness of the data itself. Therefore, validation is only a method of trying to decrease the number of errors in the input data.\\
There are various types of data validation, for example, range check, which verifies whether input data falls within a predefined range\cite{BibEntry2021Feb}.\\
For the purposes of this project, data type check is the type of validation that is necessary for validating the metadata. A data type check confirms that the data entered has the correct data type, such as a numeric field can not accept a string input.

\subsection{YAML Validation}
Validating a YAML file involves confirming the required key-value pairs. The module’s metadata can contain maps and lists, and these need to be correctly formed, to ensure proper parsing and templating of the YAML file.\\
For example, a “name” key is required to be present, and the value is required to be a string. When either of both cases is not met, the module would not be parsed or templated properly.


\section{Templating}
Templating, or web templating, is an approach to represent data in various forms. It often involves the creation of a document (template) and the presentation of data in a form that is easily understood by a human audience.\\
The template generally looks like the final form, only with placeholders instead of actual data, which are typically in a simple form.\\
Data that is presented using that template can be also separated into two parts, data required to be rendered, and data required for the template itself (navigation elements if it is a site). Combining template and data creates the final output, which is usually a web page of some kind\cite{templating}.\\
The modules in this project have also placeholders in their scripts (templates), which represent certain user-defined configurations written in the metadata file. The values of the configurations then replace the placeholders in the different scripts.\\
