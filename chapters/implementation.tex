\chapter{Implementation}
The following section discusses how the requirements were implemented. First of all, the basic structure is explained, this is followed by a detailed presentation of the code.\\
Every piece of component of the software will be implemented and explained on its own, and then the start script will be clarified. Last but not least, two examples of key modules are demonstrated.\\

\section{Structure}
Before the individual scripts in this project are explained, the general structure of the software should be overviewed.\\
As shown in figure \ref{fig:allprocess}, this work consists of multiple python components, with folders for module packages and test files.\\

\begin{itemize}
  \item \textbf{validator.py} metadata validation for modules
  \item \textbf{schema.py} schema that validator.py uses to validate against
  \item \textbf{parser.py} loads metadata and replaces configuration with placeholders
  \item \textbf{runner.py} helper functions to run modules and restore snapshots on VirtualBox
  \item \textbf{main.py} main script that automates the previous steps into one workflow
  \item \textbf{modules/} a folder that houses all implemented modules
  \item \textbf{tests/} unit tests and workflow tests for the mentioned components
\end{itemize}
The implementation of the earlier mentioned components will be explained thoroughly in the next sections.\\


\section{Validator}
Before beginning any process in this automation workflow, a validation for the meta.yml file should be prioritized, for the reason that an invalid meta.yml leads to an invalid module, since there would be a conflict of dependencies and configurations.\\

For this task, a validation tool "called “cerberus” is used. “cerberus” provides simple and lightweight data validation functionality and is designed to be easily extensible, allowing for custom validation\cite{cerberus}. It takes a pre-defined schema and validates it against the passed module’s metadata.

\begin{lstlisting}[caption=Metadata validation, style=pythonstyle]
def valid_meta(module):
    schema = eval(open('./schema.py', 'r').read())
    v = Validator(schema)
    meta = load_meta(module)
    if v.validate(meta, schema):
        return True
    else:
        # uncheck below comment to debug False validation
        # print(v.errors)
        return False
\end{lstlisting}


First and foremost, a schema python file is loaded up in “read” mode, after that, a “Validator” instance is initiated and the schema is passed on to it.\\
With a pre-defined function, the selected module’s metadata file is also loaded up to a variable in “read” mode. Lastly, a condition is set to validate this metadata against the formerly loaded schema, returning “True” when all schema rules match the metadata, and “False” when they do not.\\
\\
The schema file defines the rules of the metadata, with detailed information about the data type and requirements of every entity. A snippet from the long and very nested "schema.py" file can look like the following:
\\
\begin{lstlisting}[caption=Validation schema, style=pythonstyle]
{
# name of the module, a simple string
    'name': { 'required': True, 'type': 'string'},

    # provides: a dictionary of provided modules
    'provides': { 'required': True, 'type': 'dict', 'schema':
    {
            # dictionary of technologies that the module provides,
            # dictionary type was chosen because every tech entity can have different options/configs,
            # can be nullabe when no technologies is provided
            'tech': { 'required': True, 'type': 'list', 'schema':
            {
                'type': 'dict', 'schema': {'entry': {'type': 'dict', 'schema':
                {
                    'name': {'type': 'string'},
                    'version': {'type': 'string', 'nullable': True},
                    'config': {'type': 'list', 'nullable': True, 'schema':
                    {
                            'type': 'dict', 'schema':
                            {
                                # value of type list to contain more complex types of values
                                'name': {'type': 'string'},
                                'file': {'type': 'string'},
                                'value': {'type': 'list'}
                            }
                    }
                }}
            }}}
            },
\end{lstlisting}
The validator can be executed by running the following command:
\begin{lstlisting}[caption=Validator command, style=pythonstyle]
python validator.py [MODULE]
\end{lstlisting}
\clearpage


\section{Parser}
After validating the metadata and ensuring its usability regarding the parser, the next step in the automated process is parsing and templating. Parsing is the extraction of configuration from the metadata file, particularly the provided, as well as, the needed configuration.\\
Templating, on the other hand, as was explained in section \ref{temp_engine}, is when the template engine replaces the placeholders in the module with the given configuration in the metadata.\\ However, before starting to replace placeholders, the module package acts as a template. This means the parser first replicates the module’s folder and adds “-ready” after the name of the newly copied folder. For example, the module “ssh” becomes “ssh-ready”.\\

\begin{lstlisting}[caption=Copying directories, style=pythonstyle]
def copy_module(module):
    source = './modules/{}'.format(module)
    target = './modules/{}-ready'.format(module)
    try:
        shutil.copytree(source, target)
    except FileExistsError:
        print("Module directory already exists")
\end{lstlisting}
This allows the reuse of modules since the original module’s folder doesn’t get tampered with.\\
Placeholders are variables in bash scripts, they look like "\$\{VARIABLE\}", where VARIABLE stands for the placeholder’s name.\\ When the parser finds a configuration in the metadata, it contains the placeholder’s name, value, and the file where it should be replaced. Afterward, the parser proceeds to the mentioned file and checks for the variable, and replaces it with the given values.\\
In the snippet below \ref{templating} , the parser navigates to file path, loads it in "read" mode, and starts replacing the placeholders with "safe\_substitute()". Ultimately, the output is written in the same "ready" directory.

\begin{lstlisting}[caption=Templaing, style=pythonstyle, label=templating]
def replace(module, filename, replacements):
    script = open('./modules/{module}-ready/{filename}'.format(module=module, filename=filename), 'r')
    script_content = script.read()
    script.close()

    template = Template(script_content)
    sub = template.safe_substitute(replacements)
    outfile = open('./modules/{module}-ready/{filename}'.format(module=module, filename=filename), 'w')
    outfile.write(sub)
    outfile.close()
\end{lstlisting}

The parser can be executed by running the following command:\\
\begin{lstlisting}[caption=Parser command, style=pythonstyle]
python parser.py [MODULE]
\end{lstlisting}


\section{Interacting with VirtualBox}
The convenient “pythonic” way to communicate with Virtualbox was supposed to be a python package called “virtualbox-python”\cite{sethmlarson}. Unfortunately, the package has not been updated for quite some time, and various problems were encountered during the use of “virtualbox-python” related to locking and unlocking virtual machine’s states and sessions.\\
For this reason, a more simple way for interacting with VirtualBox was opted for, namely by directly executing bash commands from python, using the VirtualBox’s own CLI “VBoxManage”\cite{vboxmanage}. The internal package "subprocess"\cite{subprocess} provides the functionality of executing bash commands directly from python.
Next, the process of restoring a snapshot using the formerly interduced tools will be explained:
\begin{lstlisting}[caption=Restore snapshot, style=pythonstyle, escapechar=|]
def restore_snapshot(module_name, snapshot):
    process = subprocess.Popen(
        'vboxmanage snapshot {VMNAME} restore {STARTSNAPSHOT}'. \ |\label{line:vbox}|
            format(
            VMNAME=module_name,
            STARTSNAPSHOT=snapshot
        ),
        shell=True, stdout=subprocess.PIPE, stderr=subprocess.PIPE |\label{line:pipes}|
    )

    output = process.stdout.read().decode('UTF-8')
    error = process.stderr.read().decode('UTF-8')
    print(output)
    print("************") |\label{line:stars}|
    print(error)
    process.wait()
\end{lstlisting}

Initially, the function "subprocess.Popen" takes the bash command as a parameter (line \ref{line:vbox}).\\
The requested VM name and the snapshot are passed on to the command to be executed.
Also, this function captures the standard output and error which the command produced (line \ref{line:pipes}).\\
The output and error are then decoded and saved into variables to be printed after the command is done executing. The printed stars on line \ref{line:stars} are only to differentiate between the logs of the output and the error on the console.


However, one disadvantage to this method is the output and error logs are only printed out after the bash script finishes executing, which means modules with a big number of dependencies can take a decent amount of time without seeing the install log or progress in real-time on the console.\\


\section{Start Script}
Last but not least, the main script that automates all the previous components.\\
In addition to that, it has a simple dependency management to check that every module on the install queue has the needed dependencies to be able to install and/or function.\\
Nonetheless, the main script has two ways to execute depending on the user’s use case

\subsection{Multi-module mode}
\begin{lstlisting}[caption=Multi-module command, style=pythonstyle]
python main.py
\end{lstlisting}

This mode is the standard multi-module automated and managed way to install a list of modules while undertaking dependency management.\\
It starts with listing all modules in the “modules/” folder on the console while also listing every provided and needed dependency for each listed module, this gives the user an overview and brief understanding of every module and can queue them in the proper order for the install process.\\
Nonetheless, the metadata validation check occurs first when the user adds a module to the install list. If the validation is valid, the program goes on to dependency checking, and if the validation is invalid, the process is stopped and a corresponding error message is printed.\\
Dependency management, on the other hand, kicks in when the user tries to install a module, which needs one or more dependencies that are not present in the general dependencies list.
However, if that is not the case, the dependencies of the selected module are appended to the former mentioned dependencies list, to be checked again with the next module.\\
And as might be expected, when a module does not have any needed dependencies, it passes dependency management and gets appended directly to the install list.\\
After the user selects the desired modules, the install process can then be started with the keyword “run”, the program then prompts the user to enter the VM’s name and snapshot’s name, so the modules can be installed on the intended VM and the right snapshot.\\
Eventually, when the whole list of modules has finished installing, the “ready” folders of the modules are then deleted.
\\
\\

\subsection{One module mode}
\begin{lstlisting}[caption=One module command, style=pythonstyle]
python main.py -m [MODULE] -v [VM] -s [SNAPSHOT]
\end{lstlisting}

This mode is intended for more advanced use cases, as the selected module does not go through dependency checking, for the reason that there is no dependency list to compare to.\\
One module mode lets the user install one module directly from the CLI, and it is the user’s responsibility to be aware of the module’s dependencies and what it needs and provides.\\
It is activated when python recognizes any arguments after calling “main.py”, and this can be observed in upcoming snippet\ref{if_main}:

\begin{lstlisting}[caption=One module function, style=pythonstyle, label=if_main]
if __name__ == '__main__':
    if len(sys.argv) > 1:
        one_module_mode()
    else:
        multi_module_mode()
\end{lstlisting}

One module mode accepts 3 arguments, that are all required and case sensitive:
\begin{itemize}
  \item -m, --module : Name of the module
  \item -v, --vm : Name of virtual machine
  \item -s, --snapshot : Name of snapshot
\end{itemize}


It was noticed that entering an invalid snapshot name does not abort the install process, rather installs the module without restoring a snapshot.\\
However, an invalid module or VM name results in not starting the install process.

\clearpage


\section{Module Examples}
The last section in the implementation chapter will glimpse through some of the uncomplicated implemented modules, that will give a simple example of what modules could be potentially created.

\subsection{ssh}
"ssh" is an essential module to this software, as it initiates an ssh connection between the host system and the virtual machine for other modules to be able to execute scripts on the VM.

\begin{lstlisting}[caption=ssh metadata, style=pythonstyle, label=ssh]
---
name: ssh

provides:
  tech:
    - entry:
        name: ssh
        version:
        config:
            - name: PORT_FORWARDING
              file: main.sh
              value:
                - "2200"

needs:
  tech:
\end{lstlisting}

As shown in the snippet above \ref{ssh}, this represents the meta.yml file of the module "ssh", it provides only one dependency with its only configuration.\\
This configuration is "PORT\_FORWARDING", which forwards an ssh port to the host system, and the port value is “2200”. In addition, the meta.yml file indicates that the \$\{PORT\_FORWARDING\} placeholder can be found in "main.sh" file and should only be replaced there.\\ Needless to say, the module does not need any additional dependencies, therefore it passes the dependency management process.

Next, the “main.sh” script of this module will be recapped and thoroughly explained:


\clearpage

\begin{lstlisting}[caption=ssh main.sh, style=pythonstyle, escapechar=|]
!/usr/bin/env bash
export TMPDIR=modules/ssh/temp/

vboxmanage modifyvm ${VMNAME} --natpf1 "SSH,tcp,,${PORT_FORWARDING},,22" |\label{line:natpf}|
vboxmanage startvm ${VMNAME} --type headless
echo "Waiting for VM to come up..."
sleep 8

rm -f ${TMPDIR}/rootkey* |\label{line:remove_keys}|
ssh-keygen -b 2048 -t rsa -f ${TMPDIR}/rootkey -q -N ""
sshpass -p 1234 ssh-copy-id -i ${TMPDIR}/rootkey.pub -p ${PORT_FORWARDING} root@127.0.0.1
chmod 700 ${TMPDIR}/rootkey*
ssh -p ${PORT_FORWARDING} -i ${TMPDIR}/rootkey root@127.0.0.1 "ls"

vboxmanage controlvm ${VMNAME} acpipowerbutton |\label{line:powerdown}|
sleep 5
\end{lstlisting}

At the start, the temporary directory is defined to store the ssh keys that will be created.
In line \ref{line:natpf}, the “vboxmanage” API modifies the NAT rules of the virtual machine with the argument “--natpf1”. This makes the VM port forwards the value at the placeholder \${PORT\_FORWARDING}, which will be replaced by the parser to include the desired value in the metadata.

The module then starts the virtual machine in headless mode and waits for the VM to fully boot up.\\
Starting at line \ref{line:remove_keys}, the script force removes any previously created root keys that are already created in the temporary directory. Afterward, an RSA key is generated using “ssh-keygen” command and stored again in the temporary directory.\\
Next, an ssh connection is made using the public key, and again, through the previously defined forwarded port. Following this, the access permissions for the keys are modified, so other modules can also initiate a connection as the same root user.\\
Last but not least, at line \ref{line:powerdown}, a test ssh connection is made with the root key, and a “ls” command is executed, which lists all files in the current working directory.\\
Finally, the virtual machine is powered down to apply the changes and be ready for the next modules.


\clearpage

\subsection{ssh-userpass}
The purpose of this module is to automatically create users with their passwords on the Linux VM, this allows for quick generation of multiple users, for example, this can be handful for the training for privilege escalation.

\begin{lstlisting}[caption=ssh-userpass metadata, style=pythonstyle]
---
name: ssh-userpass

provides:
  tech:
    - entry:
        name: ssh
        version:
        config:
          - name: PORT_FORWARDING
            file: main.sh
            value:
            - "2200"
          - name: username:password
            file: userpass_script.sh
            value:
            - nader:pass
            - max:muster

needs:
  tech:
\end{lstlisting}

As observed in the “metadata.yml” file above, this module has the same provided features as “ssh” module, in addition to a “username:password” configuration.\\
This configuration provides a rapid way to add a user-defined list of users and passwords. It is specially handled by the parser, and it is triggered when a colon ( : ) is present in the configuration’s name.\\
The special handler is to be seen in “parser.py” script and showcased in the following snippet:


\begin{lstlisting}[caption=ssh-userpass parsing, style=pythonstyle, escapechar=|]
def user_pass_parse(module, filename, conf):
    users = conf["value"]
    for user in users:
        username, password = user.split(":")
        script = open('./modules/{module}-ready/resources/{filename}'.format(module=module, filename=filename), 'a') |\label{line:help_script}|
        script.write('sudo useradd -p $(openssl passwd -1 {password}) {username}\n'.format(username = username, password = password)) |\label{line:oneliner}|

    script.write('rm /root/userpass_script.sh')
    script.close()
\end{lstlisting}

As mentioned, when the special parser is triggered, the list of users from the metadata is passed on to the function “user\_pass\_parse()”. The users are then put through a for loop to do several tasks.\\
Firstly, the usernames and passwords are split into different variables, since they are in the form of “username:password”. Next, in line \ref{line:help_script}, a new helper script is created (userpass\_script.sh).\\
Afterward, the usernames and passwords are passed to a bash script, that adds a user to a Linux system in one line\ref{line:oneliner}. Lastly, the helper script is deleted.\\

The produced helper script will look like the following after appending the list of usernames and passwords:

\begin{lstlisting}[caption=userpass\_script.sh, style=pythonstyle]
sudo useradd -p \$(openssl passwd -1 pass) nader
sudo useradd -p \$(openssl passwd -1 muster) max
rm /root/userpass_script.sh
\end{lstlisting}
