\chapter{Introduction}
This introductory chapter provides a summary of the motivation, the desired aim, and the structure of this work.

\section{Motivation}
Getting into the world of penetrating testing is a big challenge, especially when it comes to applying what was learned theoretically to an actual machine.\\
New learners should not apply their knowledge to real targets, and setting up a testing environment can be a daunting and time-consuming task, but a program that automates this process can lift this obstacle, and with the power of Python and Bash scripts, configuring a virtual machine for pen testing can be turned into a straightforward and effortless process.

\section{Objective}
This work aims to create a simpler way to set up a virtual machine for penetration testing, in addition, it is intended to enable the user to pass a particular set of configurations through metadata.\\
This could be achieved by implementing a python script that installs a user-defined list of modules to a specific virtual machine.

\section{Approach and Structure}
This thesis can be divided into five main chapters. In the beginning, the challenges that led and inspired this work are introduced and illustrated. Chapter 2 gives an overview of the basics to understand the methods and techniques of the work, Then, in Chapter 3, the conception and design of the intended software are established.\\
Afterward, a detailed explanation of the implementation and the structure of the designed program is provided in Chapter 4. Lastly, chapter 5 gives a brief rundown on the tests and the evaluation of the development process, this gets concluded with a summary and potential future development.
